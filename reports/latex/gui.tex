\subsection{GUI}
\subsubsection{Giải thích về giao diện xử lý văn bản bằng Streamlit}

Đoạn mã này tạo ra một ứng dụng giao diện người dùng sử dụng \textbf{Streamlit} để thực hiện các tác vụ tiền xử lý văn bản tiếng Việt. Dưới đây là giải thích chi tiết cho từng phần của mã:

\begin{itemize}
    \item \textbf{Thư viện sử dụng}:
    \begin{itemize}
        \item \texttt{requests}: Được sử dụng để gửi các yêu cầu HTTP tới API.
        \item \texttt{json}: Được sử dụng để xử lý và phân tích dữ liệu JSON từ API.
        \item \texttt{streamlit}: Một thư viện phổ biến để xây dựng giao diện người dùng dễ dàng và nhanh chóng.
    \end{itemize}

    \item \textbf{API Call}:
    Hàm \texttt{call\_api} được sử dụng để gọi API từ backend. Hàm này gửi yêu cầu GET tới endpoint của API với các tham số đầu vào như văn bản cần xử lý và các lựa chọn cho các tác vụ tiền xử lý (như loại bỏ HTML, ký tự đặc biệt, khoảng trắng thừa, chuẩn hóa Unicode, dấu câu, v.v.). 
    Kết quả của việc tiền xử lý được nhận lại dưới dạng JSON và trả về dưới dạng dữ liệu đã xử lý.

    \item \textbf{Giao diện người dùng}:
    Hàm \texttt{gui} xây dựng giao diện người dùng cho việc xử lý văn bản. Giao diện bao gồm:
    \begin{itemize}
        \item Một ô nhập liệu để người dùng nhập đoạn văn bản cần xử lý.
        \item Các tùy chọn checkbox cho phép người dùng lựa chọn các tác vụ tiền xử lý, được chia thành ba cột:
        \begin{itemize}
            \item \textbf{Các tác vụ làm sạch dữ liệu}: Xóa HTML, ký tự đặc biệt, khoảng trắng thừa, và từ dừng (stopwords).
            \item \textbf{Các tác vụ chuẩn hóa dữ liệu}: Chuẩn hóa Unicode, chuyển chữ thường, chuẩn hóa dấu thanh và dấu câu.
            \item \textbf{Các tác vụ tách dữ liệu}: Tách từ và tách câu.
        \end{itemize}
        \item Nút \texttt{Xử lý}: Khi nhấn nút này, các tác vụ được thực hiện dựa trên các lựa chọn người dùng và kết quả văn bản sau khi xử lý được hiển thị bên dưới.
    \end{itemize}

    \item \textbf{Kết quả}: 
    Sau khi người dùng nhấn nút "Xử lý", văn bản đã được tiền xử lý sẽ hiển thị trong khung dưới dạng danh sách từng câu hoặc đoạn văn bản đã qua các bước xử lý mà người dùng đã lựa chọn.
\end{itemize}

\subsubsection{Code giao diện xử lý văn bản sử dụng Streamlit}

Đoạn mã dưới đây tạo ra giao diện người dùng cho việc tiền xử lý văn bản tiếng Việt sử dụng \texttt{Streamlit}. Giao diện cho phép người dùng nhập văn bản và lựa chọn các tác vụ xử lý như loại bỏ HTML, ký tự đặc biệt, khoảng trắng thừa, chuẩn hóa Unicode, chữ thường, dấu thanh, dấu câu, tách từ, tách câu và loại bỏ stopwords. Khi người dùng nhấn nút "Xử lý", kết quả sẽ được hiển thị trên giao diện.

\begin{verbatim}
import requests
import json
import streamlit as st

base_url = "http://0.0.0.0:8000/"
path_url = "api/v1/pre-processing"
headers = {"Content-Type": "application/json"}

def call_api(text,
            clean_html=True,
            clean_special_char=True,
            clean_extra_whitespace=True,
            chuan_hoa_unicode_action=True,
            chuan_hoa_chu_thuong= True,
            chuan_hoa_dau_thanh=True,
            chuan_hoa_dau_cau=True,
             split_word=True,
             split_sent=True,
             remove_sw=True
             ):
    param = {
        "text": text,
        "clean_html": clean_html,
        "clean_special_char": clean_special_char,
        "clean_extra_whitespace": clean_extra_whitespace,
        "chuan_hoa_unicode_action": chuan_hoa_unicode_action,
        "chuan_hoa_chu_thuong": chuan_hoa_chu_thuong,
        "chuan_hoa_dau_thanh": chuan_hoa_dau_thanh,
        "chuan_hoa_dau_cau": chuan_hoa_dau_cau,
        "split_word": split_word,
        "split_sent": split_sent,
        "remove_sw": remove_sw,
    }
    url = base_url + path_url
    rsp = requests.get(url=url, params=param, headers=headers)
    data = json.loads(rsp.text)["data"]
    return data

def gui():
    st.title("ĐỒ ÁN XỬ LÝ VĂN BẢN TIẾNG VIỆT")

    text = st.text_input("Nhập vào đoạn văn bản cần xử lý:")
    st.write("Lựa chọn các tác vụ bạn mong muốn:")
    col1, col2, col3 = st.columns(3)
    with col1:
        st.markdown("<p style='color: blue;'>Các tác vụ làm sạch dữ liệu</p>", unsafe_allow_html=True)
        clean_html = st.checkbox("Xoá mã HTML")
        clean_special_char = st.checkbox("Xoá các kí tự đặc biệt")
        clean_extra_whitespace = st.checkbox("Xoá các khoảng trắng thừa")
        remove_sw = st.checkbox("Loại bỏ stopwords")
    with col2:
        st.markdown("<p style='color: blue;'>Các tác vụ chuẩn hoá dữ liệu</p>", unsafe_allow_html=True)
        chuan_hoa_unicode_action = st.checkbox("Chuẩn hoá unicode")
        chuan_hoa_chu_thuong = st.checkbox("Chuẩn hoá về chữ thường")
        chuan_hoa_dau_thanh = st.checkbox("Chuẩn hoá dấu thanh")
        chuan_hoa_dau_cau = st.checkbox("Chuẩn hoá dấu câu")
    with col3:
        st.markdown("<p style='color: blue;'>Các tác vụ tách dữ liệu</p>", unsafe_allow_html=True)
        split_word = st.checkbox("Tách từ")
        split_sent = st.checkbox("Tách câu")
    if st.button('Xử lý'):
        data = call_api(text=text,
                        clean_html=clean_html,
                        clean_special_char=clean_special_char,
                        clean_extra_whitespace=clean_extra_whitespace,
                        chuan_hoa_unicode_action=chuan_hoa_unicode_action,
                        chuan_hoa_chu_thuong=chuan_hoa_chu_thuong,
                        chuan_hoa_dau_thanh=chuan_hoa_dau_thanh,
                        chuan_hoa_dau_cau= chuan_hoa_dau_cau,
                        split_word=split_word,
                        split_sent=split_sent,
                        remove_sw=remove_sw)
        with st.container():
            st.header("Văn bản sau khi được xử lý: ")
            for i in data:
                st.write(i)
    else:
        st.write('Hãy nhấn nút Xử lý để bắt đầu')
\end{verbatim}
